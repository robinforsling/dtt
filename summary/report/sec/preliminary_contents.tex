
% --- FOREWORDS ---
\newpage
\section*{Preface}

This technical report is a summary, mainly dedicated to system engineers and other practitioners, of the thesis \cite{Forsling2023Phd}. 
\begin{quotation}
	\noindent
	R. Forsling, ''The Dark Side of Decentralized Target Tracking: Unknown Correlations and Communication Constraints'', Dissertations. No. 2359, Linköping University, Linköping, Sweden, Nov. 2023.
\end{quotation}
The thesis \cite{Forsling2023Phd} is the result of my five years as an industrial PhD student, starting in 2018 and ending in 2023, at the Division of Automatic Control at Linköping University. During this time, I have been employed at Saab Aeronautics in Linköping, where I have been working with decision support systems and tactical autonomy.

We deal with optimal dimension reduction. I sometimes get the question \emph{which of the theory and algorithms in the thesis have been, or are to be, implemented in specific products?}. In my opinion, this point of view misses the main motivation for an industrial PhD project. As somebody wise once told me, ''An industrial PhD student does not return to the industry with a thesis, but rather in a pair of shoes''. The point is that it is not essential that the PhD student develops solutions to specific problems that must be addressed in the design of a certain product. What is instead important is that the PhD student returns with knowledge and expertise in a domain crucial to the company. Of course, it is beneficial if the PhD student can use a problem formulation derived from the industry during the PhD studies. However, the most important part is the academic work, where the PhD student is educated to become a standalone researcher rather than a top system developer. The latter can be done when the PhD studies are finished.

One main goal of the report is to provide the practitioner with intuition and guidelines on how to reason during the design and evaluation of target tracking systems deployed in network-centric operations. Network-centric operations in this context primarily refer to data fusion and target tracking in distributed and decentralized sensor networks. Two aspects of network-centric target tracking are addressed: (i) fusion of estimates that might be correlated to an unknown degree; and (ii) managing a constrained communication resource. I will try to be brief, and hence not all the contents of the thesis are considered. I will also try to be light in the theoretical and mathematical sense. However, some basic mathematics is required for this report to be useful.